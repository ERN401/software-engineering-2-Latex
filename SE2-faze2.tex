\documentclass[a4paper, 12pt]{report}
\usepackage{graphicx}
\usepackage{xcolor,colortbl}
\usepackage{multicol}
\usepackage{breqn,xy}
\usepackage{graphicx}
\usepackage{graphics}
\usepackage{tikz}
\usepackage{amsmath, amssymb, amsfonts, amsthm}
\usetikzlibrary{shapes.geometric,arrows}
\usepackage[a4paper,top=2cm, bottom = 2cm]{geometry}
\usepackage{indentfirst}
\usepackage[hidelinks=true, bookmarksdepth=3]{hyperref}
\usepackage{xepersian}
\SepMark{-}
\settextfont[Scale=1.1]{XB Niloofar}
\setdigitfont{Yas}
\begin{document}
	\pagenumbering{arabic}
	\begin{center}
		\thispagestyle{empty}
%		*******************************************************************
		\vspace*{1cm}
		\begin{figure}
			\begin{center}
				\includegraphics[width=4cm, height=4cm]{UniversityLogo}
			\end{center}
		\end{figure}
%		*******************************************************************
		\vspace{-1cm}
		\Large{
			\textbf{دانشگاه اصفهان \lr{-} پردیس خوانسار \\
				دانشکده ریاضی و کامپیوتر \\
			}
		}
%		*******************************************************************
		\vspace{2cm}
		\LARGE{
			گزارش فاز اول پروژه درس مهندسی نرم افزار \\
				عنوان پروژه: \textbf{سیستم مدیریت رستوران}\\
		}
%		*******************************************************************
		\vspace{2cm}		
		\Large{
			تهیه کنندگان:\\
			\textbf{
			عرفان ریاحی\\
			علی کشوری\\
			محمد مهدی هاشمی\\
			}
		}
%		*******************************************************************
		\vspace{2cm}		
		\Large{
			سرگروه: \textbf{عرفان ریاحی}\\
			نام استاد: \textbf{دکتر فضیلت حججی}\\
			نیمسال تحصیلی: \textbf{1399 \lr{-} 1400}\\
			نگارش شده با\textbf{ \LaTeX}
		}
	\end{center}
	\tableofcontents
	\chapter{مقدمه}
	\large
	در این فصل به تبیین نیازمندی‌های نرم افزار پرداخته‌ایم و با توجه به نیازمندی‌ها نرم افزار موردنظررا برای یک رستوران جهت سفارش غذا در بستر ویندوز فرم طراحی می‌کنیم.
		
%*****************************************************
	
	\chapter{معرفی پروژه}
	سامانه مدیریت رستوران(سمر) وظیفه ثبت سفارشات غذا و صدور فاکتور و دادن لیست غذاهای سفارش داده شده به سرآشپز و گارسون را دارد و همچنین از دیگر وظایف آن ارائه گزارشات مالی شامل میزان سود و زیان و نظرسنجی درمورد کیفیت غذاهای رستوران و میزان رضایتمندی مشتریان از رفتار پرسنل در قالب گزارشات ماهیانه به مدیر رستوران ارائه می‌دهد. 
	
%*****************************************************

	\chapter{آمار فاز دوم پروژه}
	
	در فاز دوم پروژه بیشتر به طراحی و برنامه نویسی وب سایت پرداخته شد.
	همچنین موضوعاتی از جمله طرح تضمین کیفیت نرم افزار، آزمون های موردنظر برای تست وب سایت و مدیریت پیکربندی وب سایت مورد بررسی قرار گرفت.\\
	
	در چندین جلسه اول فاز دوم به طراحی وب سایت پرداخته شد که این طراحی در طول پروژه دستخوش تغییراتی شد. این تغییرات در محیط Github اعمال میشد تا درصورت نیاز به نسخه‌هایی که قبلا طراحی کرده بودیم دسترسی داشته باشیم.\\
	
	چندین جلسه شامل نوشتن مباحث مربوط به طرح تضمین کیفیت نرم افزار شد. در این مباحث از طرح MaCalls استفاده شد که خود دارای سه بخش 1.تغییر و اصلاح(Revision)، 2.انتقال و ارتباط با سایر سیستم ها(Transition) و 3.عملکرد(Operation) می‌شود.\\
	
	در جلسات بعدی به انجام آزمون های واحد و انسجام پرداخته شد. \\
	
	آزمون واحد \lr{(unit testing)} روشی است برای آزمایش کردن قسمت های واحدی از پروژه تا مشخص شود که به طور صحیح کار مورد نظر ما را انجام می دهد. آزمون واحد مربوط به برنامه نویسان می‌باشد و ربطی به کاربران ندارد. 5 آزمون برای این بخش صورت گرفت که در بخش 6-1 به آن پرداخته شده است.
	آزمون انسجام \lr{(Integration Testing)} به عنوان نوعی تست تعریف می‌شود که در آن ماژول‌های نرم‌افزاری به صورت Logical یکپارچه شده و به عنوان یک گروه تست می‌شوند که در این پروژه 2 مورد آزمون صورت گرفت که در بخش 6-2 به آن پرداخته شده است.\\
	
	توضیحات مربوط به پیکربندی نرم افزار نیز در فصل 7 ذکر شده است.
		
%*****************************************************
	
	\chapter{برنامه زمان بندی پروژه}	
	
	\noindent\textbf{فاز طراحی بصری و گرافیکی}
	
	بر اساس نیازمندهای شناسایی شده در فاز اول، طراحی های لازم طی 2 هفته صورت گرفت و سپس بخش رابط کاربری طراحی و پیاده سازی شد.\\	
	
	\noindent\textbf{همگام سازی طراحی ها و رابط کاربری}
	
	پس از پیاده سازی رابط کاربری آن را با نیازمندی ها مقایسه کرده و تطابق دادیم و همگام سازی رابط کاربری و نیازمندی ها را در طول 3 روز انجام دادیم.\\
	
	\noindent\textbf{انجام تست کنترل کیفیت}
	
	با توجه به معیارهای ذکر شده در کتاب، معیارهای کیفی در وب سایت سنجیده شد و کیفیت قسمت های مختلف وب سایت مورد بررسی قرار گرفت.\\
	
	\noindent\textbf{انجام آزمون واحد و انسجام}
	
	آزمون هایی برای تست بخش های مختلف وب سایت در کتاب ذکر شده بود که 2 آزمون واحد و انسجام انجام شد و نتیجه در فصل 6 یادداشت گردید.\\
	
	
	\noindent\textbf{انجام تغییرات موردنظر پس بازبینی نهایی}
	
	پس از اصلاحات نهایی وب سایت دوباره مورد بررسی قرار گرفت و قرار شد اصلاحات لازم طی یک هفته روی وب سایت اعمال شود.
		
%*****************************************************

	\chapter{طرح تضمین کیفیت}	
	
	ما در ادامه سعی داریم تا برای این پروژه از طرح MaCalls استفاده کنیم.
	طرح MaCalls به سه زیرشاخه اصلی 1.تغییر و اصلاح(Revision)، 2.انتقال و ارتباط با سایر سیستم ها(Transition) و 3.عملکرد(Operation) تقسیم می‌شود که در ادامه به تشریح فاکتور های تضمین کیفیت در این سه بعد می‌پردازیم.\\
	
	\textbf{Revision}
	
	1.\lr{Maintainability}: این پروژه می‌بایست در زمینه اصلاح و نگهداری به خوبی عمل کند، بدین منظور می‌بایست که از سبک برنامه نویسی شئ‌گرا استفاده کرد و قابلیت های اصلاح بانک اطلاعاتی از طریق داخل برنامه به مدیر ارشد داده شود.\\
	
	2.\lr{Flexibility}: این پروژه می‌بایست که قابلیت شخصی سازی منو و واسط کاربری از جمله تِم رنگی را داشته باشد. به اینصورت که هر شخص بتواند داشبورد خود را طبق سلیقه خود بچیند و از دیگر قابلیت های آن این است که باید بتواند گزارش های خودکاری در قالب نمودار و داده در اکسل تولید کند تا مدیران بتوانند میزان رضایتمندی از محصولات خود را مشاهده کنند و میزان سود و زیان خود را در هر بخش به صورت مجزا تماشا کنند. از دیگر گزارشاتی که این تیم ارائه می‌دهد، ارائه فاکتور به کاربران می‌باشد.\\
	
	3.\lr{Testability}: این سیستم باید به گونه ای طراحی شود که کاربرد هر تابع را بتوان به طور مجزا آزمون کرد. همچنین باید از پیچیدگی ها خودداری نمود تا زمان مورد نیاز برای تست کاهش یابد.\\
	
	\textbf{Transition} 
	
	1.\lr{Portability}: در طراحی این پروژه می بایست طراحی responsive انجام گیرد و در مباحثی مانند انیمیشن هایی که در css وجود دارد برای مرورگرهای مطرح مانند Firefox ، Opera و Chrome کد مربوطه نوشته شود تا پروژه بر روی مرورگری به درستی کار کند .\\
	
	2.\lr{Reusability}: در طراحی این پروژه می بایست در صفحه هایی مانند login ، ثبت سفارش، توابع محاسبه سود و زیان، فراموشی رمز عبور و داشبورد، فایل های dll تهیه نمود تا بتوان از آن در پروژه‌های دیگر نیز استفاده شود همچنین در طراحی می بایست به گونه ای عمل شود که بتوان از این پروژه در دیگر پروژه ها استفاده کرد، به این گونه که بتوان منو و در صورت نیاز بتوان نوع فروش کالا را تغییر داد .\\
	
	3.\lr{Interoperability}: این سیستم می‌بایست به خوبی با سیستم هایی همچون درگاه های پرداخت و بانک اطلاعاتی ارتباط برقرار کند.\\
	
	\textbf{Operation}
	
	1.\lr{Correctness}: در طراحی این پروژه می‌بایست از دقیق بودن فرمول های محاسبه میزان حقوق و دستمزد، میزان رضایتمندی و میزان سود و زیان اطمینان حاصل نمود.\\
	
	2.\lr{Usability}: از آنجایی که استفاده صحیح از برنامه در طول عمر آن تأثیر دارد، می‌بایست آموزشی در این رابطه تهیه و به کارفرمایان و کارگران ارائه شود.\\
	
	3.\lr{Efficiency}: این سیستم می‌بایست در هر روز به مدت ۱۰ ساعت فعال باشد و از آنجایی که پیک غذا خوردن به هنگام ناهار و شام می باشد، باید به گونه ای سیستم را طراحی کرد که توانایی پردازش موازی را داشته باشد. همینطور می‌بایست در تهیه هاست توجه داشت که رستوران در روز چقدر مشتری داشته است و با توجه به داده‌های به دست آمده هاست مناسب پیشنهاد بدهیم .\\
	
	4.\lr{Reliability}: باید اطمینان حاصل پیدا کرد که تیم پشتیبانی آماده باشد و پشتیبان گیری هایی به موقع از سیستیم این اطمینان را به کاربران بدهیم که اطلاعات آنها از بین نخواهد رفت و همین طور می بایست در صورت استفاده از سرور های شخصی از وجود برق پشتیبان اطمینان پیدا کرد تا سیستم در کمترین زمان ممکن درصورت غیرفعال شدن، مجدداً فعال شود .\\
	
	5.\lr{Integrity}: با استفاده از dll های موجود می بایست اطمینان حاصل کنیم که سیستم از امنیت بالایی برخوردار است تا هکرها نتوانند به اطلاعات کاربران دسترسی داشته باشند، به این منظور می بایست از هکرهای سفید کمک گرفت تا تمام نقاط ضعف سیستم شناسایی شود.
	
	\chapter{آزمون نرم افزار}
	\section{آزمون واحد}
	
	1. ثبت نام/ورود
	\begin{table}[h]
		\begin{center}
			\resizebox{\textwidth}{!}{
				\begin{tabular}{|c|c|c|c|c|}
					\rowcolor{blue!40!white}
					\textbf{\textcolor{white}{\#}} & \textbf{\textcolor{white}{فاعل}} & \textbf{\textcolor{white}{کنش فاعل}} & 
					\textbf{\textcolor{white}{دیگرداده‌ها/اشیا}} & \textbf{\textcolor{white}{شیئی که کنش روی آن انجام می‌شود}} \\
					
					\rowcolor{black!10}
					1 & کاربر & کلیک بر روی دکمه ثبت نام/ورود & بازکردن صفحه login & سیستم\\
					\hline
					
					1.2 & کاربر & ورود نام کاربری و رمزعبور & فیلد نام کاربری و رمز عبور & سیستم\\
					\hline
					
					\rowcolor{black!10}
					1.3 & کاربر & کلیک بر روی ورود & ارسال اطلاعات & اطلاعات \\
					\hline
					
					1.4 & سیستم & می‌فرستد & اطلاعات & \lr{DB manager} \\
					
					\hline
					
					\rowcolor{black!10}
					1.5 & \multicolumn{4}{|c|}{اگر اطلاعات دریافت شده در \lr{DB manager} موجود بود،} \\
					\hline										
					
					1.5.1 & سیستم & نشان می‌دهد & داشبورد کاربر & کاربر \\
					\hline
					
					\rowcolor{black!10}
					1.5.2 & \multicolumn{4}{|c|}{در غیر اینصورت،} \\
					\hline
					
					1.5.3 & سیستم & نشان دادن پیام & پیام & کاربر \\
					\hline										
				\end{tabular}
			}
		\end{center}
		\caption{سناریو ثبت نام/ورود}
	\end{table}
	
	2. فراموشی رمز عبور
	\begin{table}[h]
		\begin{center}
			\resizebox{\textwidth}{!}{
				\begin{tabular}{|c|c|c|c|c|}
					\rowcolor{blue!40!white}
					\textbf{\textcolor{white}{\#}} & \textbf{\textcolor{white}{فاعل}} & \textbf{\textcolor{white}{کنش فاعل}} & 
					\textbf{\textcolor{white}{دیگرداده‌ها/اشیا}} & \textbf{\textcolor{white}{شیئی که کنش روی آن انجام می‌شود}} \\
					
					\rowcolor{black!10}
					1 & کاربر & کلیک بر روی دکمه ثبت نام/ورود & بازکردن صفحه login & سیستم\\
					\hline
					
					2 & کاربر & کلیک بر روی دکمه فراموشی رمز عبور & بازکردن صفحه forget & سیستم\\
					\hline
					
					\rowcolor{black!10}
					3 & کاربر & ورود ایمیل و شماره تلفن & فیلد اطلاعات & سیستم \\
					\hline
					
					4 & کاربر & بر روی ارسال کد تایید کلیک می‌کند & ارسال اطلاعات & سیستم\\
					
					\hline
					
					\rowcolor{black!10}
		5 & سیستم & می‌فرستد & اطلاعات & \lr{DB manager} \\
					\hline
					
					5.1 & \multicolumn{4}{|c|}{اگر اطلاعات در \lr{DB manager} موجود بود،} \\
					\hline										
					
		5.1.2 & سیستم & می‌فرستد & کد بازیابی & ایمیل کاربر \\
					\hline
					
					\rowcolor{black!10}
		5.1.3 & کاربر & وارد می‌کند & کد دریافتی & در سیستم \\
					\hline
					
	5.1.4 & کاربر & وارد می‌کند & رمز عبور جدید & فیلد \\
					\hline
		
		\rowcolor{black!10}
		5.1.5 & کاربر & کلیک می‌کند & دکمه ثبت & سیستم \\
		\hline
		
5.1.6 & سیستم & می‌فرستد & رمز عبور جدید & \lr{DB manager} \\
		\hline		
					
					\rowcolor{black!10}
					5.2 & \multicolumn{4}{|c|}{در غیر اینصورت،} \\
					\hline
					
			5.2.1 & سیستم & نمایش می‌دهد & پیام & کاربر \\
					\hline										
				\end{tabular}
			}
		\end{center}
		\caption{سناریو فراموشی رمز عبور}
	\end{table}
	\vspace{5cm}
	
	3. ثبت نام
	\begin{table}[h]
		\begin{center}
			\resizebox{\textwidth}{!}{
				\begin{tabular}{|c|c|c|c|c|}
					\rowcolor{blue!40!white}
					\textbf{\textcolor{white}{\#}} & \textbf{\textcolor{white}{فاعل}} & \textbf{\textcolor{white}{کنش فاعل}} & 
					\textbf{\textcolor{white}{دیگرداده‌ها/اشیا}} & \textbf{\textcolor{white}{شیئی که کنش روی آن انجام می‌شود}} \\
					
					\rowcolor{black!10}
					1 & کاربر & کلیک می‌کند & دکمه ورود/ثبت نام & سیستم\\
					\hline
					
					2 & سیستم & نمایش می‌دهد & صفحه \lr{sign up} & کاربر\\
					\hline
					
					\rowcolor{black!10}
					3 & کاربر & وارد می‌کند & اطلاعات & سیستم \\
					\hline
					
					4 & سیستم & می‌فرستد & اطلاعات & \lr{DB manager}\\					
					\hline
					
					\rowcolor{black!10}					
					4.1 & \multicolumn{4}{|c|}{اگر اطلاعات داده شده در \lr{DB manager} موجود نبود،} \\
					\hline										
					
				4.1.1 & سیستم & وارد می‌کند & اطلاعات & \lr{DB manager} \\
					\hline
					
					\rowcolor{black!10}
				4.1.2 & سیستم & نشان می‌دهد & داشبور & کاربر \\
					\hline
					
					4.2 & \multicolumn{4}{|c|}{در غیر اینصورت،} \\
					\hline
					
					\rowcolor{black!10}
					4.2.1 & سیستم & نشان می‌دهد & پیام & کاربر \\
					\hline										
				\end{tabular}
			}
		\end{center}
		\caption{سناریو ثبت نام}
	\end{table}

4. ارسال انتقادات
\begin{table}[h]
	\begin{center}
		\resizebox{\textwidth}{!}{
			\begin{tabular}{|c|c|c|c|c|}
				\rowcolor{blue!40!white}
				\textbf{\textcolor{white}{\#}} & \textbf{\textcolor{white}{فاعل}} & \textbf{\textcolor{white}{کنش فاعل}} & 
				\textbf{\textcolor{white}{دیگرداده‌ها/اشیا}} & \textbf{\textcolor{white}{شیئی که کنش روی آن انجام می‌شود}} \\
				
				\rowcolor{black!10}
				1 & کاربر & کلیک می‌کند & دکمه ارتباط با ما & سیستم\\
				\hline
				
				2 & سیستم & نشان می‌دهد & صفحه ارسال پیام & کاربر \\
				\hline
				
				\rowcolor{black!10}
				3 & کاربر & وارد می‌کند & نقد خود را & سیستم \\
				\hline
				
				4 & کاربر & کلیک می‌کند & دکمه ارسال & سیستم \\				
				\hline
				
				\rowcolor{black!10}				
				5 & سیستم & می‌فرستد & پیام & به داشبور مدیریت \\
				\hline										
			\end{tabular}
		}
	\end{center}
	\caption{سناریو ارسال انتقادات}
\end{table}

5. ثبت سفارش
\begin{table}[h!]
	\begin{center}
		\resizebox{\textwidth}{!}{
			\begin{tabular}{|c|c|c|c|c|}
				\rowcolor{blue!40!white}
				\textbf{\textcolor{white}{\#}} & \textbf{\textcolor{white}{فاعل}} & \textbf{\textcolor{white}{کنش فاعل}} & 
				\textbf{\textcolor{white}{دیگرداده‌ها/اشیا}} & \textbf{\textcolor{white}{شیئی که کنش روی آن انجام می‌شود}} \\
				
				\rowcolor{black!10}
	1 & کاربر & کلیک می‌کند & بر روی + برای اضافه کردن & غذا \\
				\hline
				
	2 & کاربر & کلیک می‌کند & بر روی - برای حذف کردن & غذا\\
				\hline
				
				\rowcolor{black!10}
	3 & سیستم & محاسبه می‌کند & قیمت & کاربر \\
				\hline
				
	4 & کاربر & کلیک می‌کند & دکمه پرداخت & سیستم \\					
				\hline
								
				\rowcolor{black!10}					
	5 & سیستم & باز می‌کند & درگاه بانک & کاربر \\
				\hline
				
	6 & کاربر & وارد می‌کند & اطلاعات کارت & سیستم \\
				\hline
				
				\rowcolor{black!10}
				6.1 & \multicolumn{4}{|c|}{اگر پرداخت موفق بود،} \\
				\hline										
				
		6.1.1 & سیستم & صادر می‌کند & فاکتور & کاربر \\
				\hline
				
				\rowcolor{black!10}				
				6.2 & \multicolumn{4}{|c|}{در غیر اینصورت،} \\
				\hline
								
	6.2.1 & سیستم & نشان می‌دهد & کد پیگیری & کاربر \\
				\hline										
			\end{tabular}
		}
	\end{center}
	\caption{سناریو ثبت سفارش}
\end{table}

\vspace{5cm}
\section{آزمون انسجام}
{\Large\textbf{آزمون 1}}

\begin{tikzpicture}[sibling distance=10em,
	every node/.style = {shape=rectangle, rounded corners,
		draw, align=center,
		top color=white, bottom color=blue!20}]]
	\node {\rl{ثبت نام}}
	child { node {\rl{فیلد اطلاعات} }
		child { node {\rl{پایگاه داده} } } }
	child { node {\rl{فراموشی رمز} }
		child { node {\rl{فیلد اطلاعات} }
			child { node {\rl{پایگاه داده} } } } }
	child { node {\rl{ورود} } 
		child { node {\rl{فیلد اطلاعات} } 
			child { node {\rl{پایگاه داده} } 
				child { node {\rl{داشبورد} } 
					child { node {\rl{ویرایش اطلاعات} } } } } } };		
\end{tikzpicture}

\vspace{2cm}
{\Large\textbf{آزمون 2}}
\begin{center}
	\tikzstyle{block} = [shape=rectangle, rounded corners,
	draw, align=center, top color=white, bottom color=blue!20]
	\tikzstyle{line} = [draw, -latex']
	\begin{tikzpicture}[node distance = 2cm]
		\node [block] (sabt) {\rl{ثبت سفارش}};
		\node [block, below of=sabt] (entekhab) {\rl{انتخاب غذا}};
		\node [block, below of=entekhab, xshift=-2cm] (+) {$ + $};
		\node [block, below of=entekhab, xshift=2cm] (-) {$ - $};
		\node [block, below of=entekhab, yshift=-2cm] (namayesh) {\rl{نمایش تعداد}};
		\node [block, below of=namayesh] (dargah) {\rl{درگاه پرداخت}};
		\node [block, below of=dargah] (factor) {\rl{مشاهده فاکتور}};
		Draw edges
		\path [line] (sabt) -- (entekhab);
		\path [line] (entekhab) -- (+);
		\path [line] (entekhab) -- (-);
		\path [line] (+) -- (namayesh);
		\path [line] (-) -- (namayesh);
		\path [line] (namayesh) -- (dargah);
		\path [line] (dargah) -- (factor);
	\end{tikzpicture}
\end{center}

\chapter{مدیریت پیکربندی}

برای مدیریت پیکربندی از دو سیستم \lr{version controller} به نام های Github و Azure استفاده می‌کنیم. سعی ما بر این است که با قراردادن ورژن های مختلف وب سایت در Github ، بر روی ورژن‌های مختلف برنامه کنترل داشته باشیم تا درصورت بروز خطا بتوانیم به ورژن های قبلی وب سایت دسترسی داشته باشیم و اصلاحات موردنیاز را در آن اعمال کنیم.\\

همچنین از محیط Azure برای کنترل و تقسیم کارها بین اعضای تیم و پیکربندی مستندات استفاده می‌کنیم.\\
 
در کد برنامه خطاهایی وجود داشت که در جلسات برگزار شده و انجام آزمون نرم افزار پی به آنها بردیم و  آنها را اصلاح کردیم. این خطاها به شرح زیر می‌باشد:

\begin{enumerate}
	\item 
تغییر UI وب سایت
	\item
	responsive کردن وب سایت
	\item	
	کار نکردن لینک مربوط به دکمه‌ی ورود	
	\item
	کار نکردن لینک مربوط به دکمه‌ی ثبت نام
	\item	
	اضافه کردن دکمه‌ی + و -
	\item
	تغییر cursor از نوع text به pointer برای دکمه‌های + و –
	\item
	ایجاد margin بین "+"، "تعداد" و "–" 
	\item
	اصلاح لینک داخلی "درباره ما"	
\end{enumerate}

\chapter{توضیحات پروژه}

\indent
در طراحی وب سایت به قسمت \lr{front-End} بسنده کردیم که در این راستا از Html ، CSS و Js استفاده کردیم.\\

در این پروژه صفحه‌ای برای ثبت نام و ورود قرار دادیم، همچنین صفحه‌ای برای فراموشی رمز عبور داشتیم.\\

در صفحه اصلی 2 دکمه برای مشاهده منو که یکی در قسمت منوی بالای سایت و دیگری در وسط صفحه به هنگام لود شدن صفحه وجود دارد.\\

دکمه‌های "درباره ما" و "ارتباط با ما" وجود دارد که با کلیک بر روی دکمه "درباره ما" اطلاعات رستوران شامل ساعات کاری و آدرس نمایش داده می‌شود و با کلیک بر روی دکمه "ارتباط با ما" قسمت مربوط به ارسال نظرات نمایش داده می‌شود.\\

با کلیک بر روی دکمه home به شکل 
\includegraphics[width=0.8cm, height=0.8cm]{home}
، در قسمت منو به ابتدای صفحه باز می‌گردیم و با کلیک بر روی دکمه ورود/ثبت نام، به صفحه ورود و ثبت نام منتقل می‌شویم.\\

در قسمت منو دکمه‌هایی جهت اضافه و کم کردن غذاها و نوشیدنی‌ها وجود دارد که با کلیک بر روی این دکمه ها، تعداد انتخابی نشان داده می‌شود.\\

در آخر هم مجموع مبلغ همه‌ی سفارش محاسبه شده و نشان داده می‌شود که با کلیک بر روی دکمه پرداخت یک alert با عنوان "فاکتور شماره 1 پرداخت شد" نمایان می‌شود.
\end{document}